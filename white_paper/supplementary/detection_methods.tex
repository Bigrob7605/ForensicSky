\section{Detailed Detection Methods}

\subsection{Modern Exotic Physics Hunter v3.0}

The Modern Exotic Physics Hunter v3.0 implements 9 specialized physics channels for cosmic string detection:

\subsubsection{Axion Oscillations}
Axion oscillations are detected using a combination of:
\begin{itemize}
    \item Fourier analysis of timing residuals
    \item Power spectral density estimation
    \item Cross-correlation analysis between pulsars
    \item Bayesian inference for parameter estimation
\end{itemize}

The detection algorithm searches for characteristic oscillation patterns with frequencies in the range $10^{-9}$ to $10^{-6}$ Hz, corresponding to axion masses between $10^{-22}$ and $10^{-19}$ eV.

\subsubsection{Axion Clouds}
Axion cloud signatures are identified through:
\begin{itemize}
    \item Gravitational wave emission analysis
    \item Orbital decay measurements
    \item Spin-down rate variations
    \item Tidal interaction effects
\end{itemize}

\subsubsection{Dark Photons}
Dark photon signatures are detected using:
\begin{itemize}
    \item Electromagnetic field coupling analysis
    \item Dispersion measure variations
    \item Polarization state changes
    \item Frequency-dependent effects
\end{itemize}

\subsubsection{Scalar Fields}
Scalar field signatures are identified through:
\begin{itemize}
    \item Fifth force interactions
    \item Modified gravity effects
    \item Spacetime curvature variations
    \item Energy-momentum tensor modifications
\end{itemize}

\subsubsection{Primordial Black Holes}
Primordial black hole signatures are detected using:
\begin{itemize}
    \item Gravitational lensing effects
    \item Microlensing light curves
    \item Astrometric deflections
    \item Timing delay variations
\end{itemize}

\subsubsection{Domain Walls}
Domain wall signatures are identified through:
\begin{itemize}
    \item Scalar field gradient analysis
    \item Energy density variations
    \item Spacetime metric perturbations
    \item Gravitational wave emission
\end{itemize}

\subsubsection{Fifth Force}
Fifth force signatures are detected using:
\begin{itemize}
    \item Modified Newtonian dynamics
    \item Gravitational constant variations
    \item Spacetime geometry changes
    \item Matter coupling modifications
\end{itemize}

\subsubsection{Quantum Gravity Effects}
Quantum gravity signatures are identified through:
\begin{itemize}
    \item Spacetime foam analysis
    \item Modified dispersion relations
    \item Quantum decoherence effects
    \item Planck-scale physics
\end{itemize}

\subsubsection{Extra Dimensions}
Extra dimension signatures are detected using:
\begin{itemize}
    \item Kaluza-Klein mode analysis
    \item Brane world effects
    \item Graviton emission
    \item Compactification radius variations
\end{itemize}

\subsection{Advanced Cosmic String Hunter}

The Advanced Cosmic String Hunter implements 5 specialized detection methods:

\subsubsection{Cusp Bursts}
Cusp burst detection uses the Damour-Vilenkin template:
\begin{equation}
h(t) = A \left(\frac{t-t_0}{\tau}\right)^{-4/3} \Theta(t-t_0)
\end{equation}
where $A$ is the amplitude, $t_0$ is the burst time, $\tau$ is the characteristic timescale, and $\Theta$ is the Heaviside step function.

\subsubsection{Kink Radiation}
Kink radiation is detected using:
\begin{itemize}
    \item Gravitational wave burst analysis
    \item Frequency spectrum fitting
    \item Polarization state determination
    \item Sky localization
\end{itemize}

\subsubsection{Stochastic Background}
The stochastic background is analyzed using:
\begin{itemize}
    \item Hellings-Downs correlation function
    \item Power spectral density estimation
    \item Cross-correlation analysis
    \item Bayesian parameter estimation
\end{itemize}

\subsubsection{Non-Gaussian Correlations}
Non-Gaussian correlations are detected using:
\begin{itemize}
    \item Higher-order moment analysis
    \item Bispectrum estimation
    \item Trispectrum analysis
    \item Non-Gaussianity tests
\end{itemize}

\subsubsection{Lensing Effects}
Lensing effects are identified through:
\begin{itemize}
    \item Magnification variations
    \item Astrometric deflections
    \item Time delay measurements
    \item Image splitting analysis
\end{itemize}

\subsection{Deep Learning Integration}

\subsubsection{Transformers}
Transformer networks are used for:
\begin{itemize}
    \item Sequence modeling of timing residuals
    \item Attention mechanism for feature extraction
    \item Long-range dependency detection
    \item Pattern recognition
\end{itemize}

\subsubsection{Variational Autoencoders (VAE)}
VAEs are employed for:
\begin{itemize}
    \item Anomaly detection
    \item Unsupervised learning
    \item Dimensionality reduction
    \item Generative modeling
\end{itemize}

\subsubsection{Graph Neural Networks}
Graph neural networks are used for:
\begin{itemize}
    \item Pulsar network analysis
    \item Spatial correlation modeling
    \item Topology analysis
    \item Relationship learning
\end{itemize}

\subsection{Quantum-Inspired Methods}

\subsubsection{Quantum Optimization}
Quantum optimization algorithms are used for:
\begin{itemize}
    \item Parameter space exploration
    \item Global optimization
    \item Constraint satisfaction
    \item Multi-objective optimization
\end{itemize}

\subsubsection{Quantum Gravity Search}
Quantum gravity search algorithms are employed for:
\begin{itemize}
    \item Spacetime structure analysis
    \item Quantum effect detection
    \item Planck-scale physics
    \item Emergent gravity
\end{itemize}

\subsection{Ensemble Bayesian Analysis}

The ensemble Bayesian analysis combines multiple detection methods using:
\begin{itemize}
    \item Bayesian model averaging
    \item Posterior probability estimation
    \item Evidence ratio calculation
    \item Model selection criteria
\end{itemize}

The final significance is calculated as:
\begin{equation}
\sigma_{\text{combined}} = \sqrt{\sum_{i=1}^{N} \sigma_i^2}
\end{equation}
where $\sigma_i$ is the significance from method $i$ and $N$ is the number of methods.
