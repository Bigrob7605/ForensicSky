\section{Data Processing Pipeline}

\subsection{IPTA DR2 Dataset}

\subsubsection{Dataset Overview}
The International Pulsar Timing Array Data Release 2 (IPTA DR2) contains timing data for 45 millisecond pulsars observed by multiple radio telescopes over more than 20 years.

\subsubsection{Observatories}
\begin{itemize}
    \item \textbf{Jodrell Bank Observatory (JBO)}: 1520 MHz observations
    \item \textbf{Nancay Radio Telescope (NRT)}: 1400, 1600, 2000 MHz observations
    \item \textbf{Effelsberg (EFF)}: 1360, 1410, 2639 MHz observations
    \item \textbf{Westerbork Synthesis Radio Telescope (WSRT)}: 1380 MHz observations
\end{itemize}

\subsubsection{Pulsar Selection}
The 45 pulsars were selected based on:
\begin{itemize}
    \item Timing precision (<1 μs RMS)
    \item Observation span (>10 years)
    \item Data quality (low RFI, stable profiles)
    \item Sky distribution (good angular coverage)
\end{itemize}

\subsection{Timing Analysis}

\subsubsection{Standard Pulsar Timing}
Timing residuals were calculated using standard pulsar timing analysis:

\begin{equation}
R(t) = t_{\text{arrival}} - t_{\text{predicted}}
\end{equation}

where $t_{\text{arrival}}$ is the observed pulse arrival time and $t_{\text{predicted}}$ is the predicted arrival time from the timing model.

\subsubsection{Timing Model}
The timing model includes:
\begin{itemize}
    \item Pulsar spin parameters (period, period derivative)
    \item Binary parameters (orbital period, eccentricity, etc.)
    \item Astrometric parameters (position, proper motion, parallax)
    \item Dispersion measure variations
    \item Solar system ephemeris
\end{itemize}

\subsubsection{Software}
Timing analysis was performed using:
\begin{itemize}
    \item TEMPO2 for timing model fitting
    \item Custom analysis scripts for cosmic string detection
    \item Python-based data processing pipeline
    \item Statistical analysis tools
\end{itemize}

\subsection{Data Quality Control}

\subsubsection{Outlier Detection}
Outliers were identified and flagged using:
\begin{itemize}
    \item 3σ clipping
    \item Median absolute deviation (MAD)
    \item Robust statistics
    \item Visual inspection
\end{itemize}

\subsubsection{RFI Mitigation}
Radio frequency interference (RFI) was mitigated through:
\begin{itemize}
    \item Frequency domain filtering
    \item Time domain analysis
    \item Cross-correlation with known RFI sources
    \item Manual flagging of obvious interference
\end{itemize}

\subsubsection{Data Validation}
Data quality was validated using:
\begin{itemize}
    \item Timing precision checks
    \item Profile stability analysis
    \item Dispersion measure consistency
    \item Cross-observatory comparisons
\end{itemize}

\subsection{Preprocessing}

\subsubsection{Detrending}
Long-term trends were removed using:
\begin{itemize}
    \item Linear detrending
    \item Polynomial fitting
    \item Spline smoothing
    \item Wavelet decomposition
\end{itemize}

\subsubsection{Filtering}
High-frequency noise was filtered using:
\begin{itemize}
    \item Low-pass filtering
    \item Band-pass filtering
    \item Wiener filtering
    \item Kalman filtering
\end{itemize}

\subsubsection{Normalization}
Data was normalized using:
\begin{itemize}
    \item Z-score normalization
    \item Min-max scaling
    \item Robust scaling
    \item Quantile normalization
\end{itemize}

\subsection{Feature Extraction}

\subsubsection{Time Domain Features}
Time domain features were extracted including:
\begin{itemize}
    \item Mean, standard deviation, skewness, kurtosis
    \item Autocorrelation function
    \item Power spectral density
    \item Wavelet coefficients
\end{itemize}

\subsubsection{Frequency Domain Features}
Frequency domain features included:
\begin{itemize}
    \item Fourier transform coefficients
    \item Power spectral density
    \item Spectral moments
    \item Frequency band analysis
\end{itemize}

\subsubsection{Statistical Features}
Statistical features were calculated:
\begin{itemize}
    \item Higher-order moments
    \item Information-theoretic measures
    \item Entropy measures
    \item Fractal dimensions
\end{itemize}

\subsection{Cross-Correlation Analysis}

\subsubsection{Pulsar Pairs}
Cross-correlations were calculated for all pulsar pairs:
\begin{equation}
C_{ij}(\tau) = \frac{1}{N} \sum_{t} R_i(t) R_j(t+\tau)
\end{equation}

where $R_i(t)$ and $R_j(t)$ are the timing residuals for pulsars $i$ and $j$, and $\tau$ is the time lag.

\subsubsection{Hellings-Downs Correlation}
The Hellings-Downs correlation function was calculated:
\begin{equation}
\zeta(\theta) = \frac{1}{2} - \frac{1}{4} \left(1 + \cos\theta\right) + \frac{3}{2} \left(1 - \cos\theta\right) \ln\left(\frac{1 - \cos\theta}{2}\right)
\end{equation}

where $\theta$ is the angular separation between pulsars.

\subsection{Noise Modeling}

\subsubsection{White Noise}
White noise was modeled as:
\begin{equation}
\sigma_{\text{white}} = \sigma_{\text{TOA}} \sqrt{\frac{1}{N_{\text{obs}}}}
\end{equation}

where $\sigma_{\text{TOA}}$ is the TOA uncertainty and $N_{\text{obs}}$ is the number of observations.

\subsubsection{Red Noise}
Red noise was modeled using a power-law spectrum:
\begin{equation}
P(f) = A^2 \left(\frac{f}{f_{\text{ref}}}\right)^{-\gamma}
\end{equation}

where $A$ is the amplitude, $f_{\text{ref}}$ is the reference frequency, and $\gamma$ is the spectral index.

\subsubsection{Systematic Noise}
Systematic noise sources included:
\begin{itemize}
    \item Clock errors
    \item Ephemeris errors
    \item Ionospheric effects
    \item Tropospheric effects
\end{itemize}

\subsection{Data Archival}

\subsubsection{Metadata}
Comprehensive metadata was recorded including:
\begin{itemize}
    \item Observation parameters
    \item Data quality flags
    \item Processing history
    \item Software versions
\end{itemize}

\subsubsection{Provenance}
Data provenance was tracked through:
\begin{itemize}
    \item Version control
    \item Change logs
    \item Audit trails
    \item Documentation
\end{itemize}

\subsubsection{Access Control}
Data access was controlled through:
\begin{itemize}
    \item User authentication
    \item Permission systems
    \item Usage logging
    \item Backup systems
\end{itemize}

\subsection{Reproducibility}

\subsubsection{Code Documentation}
All code was documented with:
\begin{itemize}
    \item Detailed comments
    \item Function descriptions
    \item Parameter explanations
    \item Usage examples
\end{itemize}

\subsubsection{Version Control}
All code was version controlled with:
\begin{itemize}
    \item Git repository
    \item Commit messages
    \item Tagged releases
    \item Change logs
\end{itemize}

\subsubsection{Containerization}
Analysis was containerized using:
\begin{itemize}
    \item Docker containers
    \item Environment specifications
    \item Dependency management
    \item Portability
\end{itemize}

\subsection{Quality Assurance}

\subsubsection{Testing}
Comprehensive testing was performed:
\begin{itemize}
    \item Unit tests
    \item Integration tests
    \item Regression tests
    \item Performance tests
\end{itemize}

\subsubsection{Validation}
Results were validated using:
\begin{itemize}
    \item Cross-validation
    \item Bootstrap analysis
    \item Monte Carlo simulations
    \item Independent verification
\end{itemize}

\subsubsection{Documentation}
Comprehensive documentation was maintained:
\begin{itemize}
    \item User manuals
    \item Technical specifications
    \item API documentation
    \item Tutorials
\end{itemize}
